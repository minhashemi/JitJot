
\subsection*{1. Expressions}
\begin{itemize}
    \item An expression is a combination of variables, operators, and values that produces a result.
    \item Example: 
    \begin{verbatim}
x = 5 + 3  
y = x * 2  
z = (x + y) / 2  
    \end{verbatim}
\end{itemize}

\subsection*{2. Binary Operations}
\begin{itemize}
    \item Operations on numbers that work at the binary level.
    \item Example:
    \begin{verbatim}
x = 5  # 101 in binary
y = 3  # 011 in binary
and_result = x & y  # 001 in binary, so result = 1
or_result = x | y   # 111 in binary, so result = 7
xor_result = x ^ y  # 110 in binary, so result = 6
left_shift = x << 1  # 1010 in binary, result = 10
right_shift = x >> 1  # 010 in binary, result = 2
    \end{verbatim}
\end{itemize}

\subsection*{3. Functions}
\begin{itemize}
    \item A function is a block of reusable code that performs a specific task.
    \item Syntax:
    \begin{verbatim}
def add(a, b):
    return a + b
    \end{verbatim}
    \item Example:
    \begin{verbatim}
def multiply(a, b):
    return a * b

result = multiply(5, 3)  # result is 15
    \end{verbatim}
\end{itemize}

\subsection*{4. Recursion}
\begin{itemize}
    \item A function that calls itself.
    \item Key components:
    \begin{itemize}
        \item \textbf{Base Case}: Prevents infinite recursion.
        \item \textbf{Recursive Case}: Reduces the problem.
    \end{itemize}
    \item Example: Factorial of a number.
    \begin{verbatim}
def factorial(n):
    if n == 0:
        return 1  # Base case
    return n * factorial(n-1)  # Recursive case

print(factorial(5))  # 120
    \end{verbatim}
\end{itemize}

\subsection*{5. Conditionals: if, else, elif}
\begin{itemize}
    \item \texttt{if}: Runs code if the condition is true.
    \item \texttt{else}: Runs code if the condition is false.
    \item \texttt{elif}: Checks additional conditions if the previous ones are false.
    \item Example:
    \begin{verbatim}
x = 10
if x < 5:
    print("x is less than 5")
elif x == 10:
    print("x is 10")
else:
    print("x is greater than 5")
    \end{verbatim}
\end{itemize}

\subsection*{6. Loops: for, while}
\begin{itemize}
    \item \texttt{for}: Iterates over a sequence (e.g., list, range).
    \item \texttt{while}: Repeats code as long as a condition is true.
    \item Example (\texttt{for} loop):
    \begin{verbatim}
for i in range(5):
    print(i)
    \end{verbatim}
    \item Example (\texttt{while} loop):
    \begin{verbatim}
count = 0
while count < 5:
    print(count)
    count += 1
    \end{verbatim}
\end{itemize}

\subsection*{7. Lists}
\begin{itemize}
    \item Ordered, mutable collection of elements.
    \item Syntax: \texttt{list = [element1, element2, element3]}
    \item Example:
    \begin{verbatim}
my_list = [1, 2, 3]
my_list.append(4)  # Adds 4 to the end
my_list[0] = 0  # Updates the first element
print(my_list)  # [0, 2, 3, 4]

# Loop through the list
for item in my_list:
    print(item)  # Prints each element

# List comprehension to square each number
squared = [x**2 for x in my_list]
print(squared)  # [0, 4, 9, 16]
    \end{verbatim}
    \item Methods:
    \begin{itemize}
        \item \textbf{\texttt{.append()}}: Add an element to the end.
        \begin{verbatim}
        lst = [1, 2]
        lst.append(3)
        lst  # Output: [1, 2, 3]
        \end{verbatim}
        \item \textbf{\texttt{.remove()}}: Remove the first occurrence of an element.
        \begin{verbatim}
        lst = [1, 2, 3]
        lst.remove(2)
        lst  # Output: [1, 3]
        \end{verbatim}
        \item \textbf{\texttt{.pop()}}: Remove and return an element by index.
        \begin{verbatim}
        lst = [1, 2, 3]
        lst.pop(1)  # Output: 2
        lst  # Output: [1, 3]
        \end{verbatim}
        \item \textbf{\texttt{.sort()}}: Sort the list in place.
        \begin{verbatim}
        lst = [3, 1, 2]
        lst.sort()
        lst  # Output: [1, 2, 3]
        \end{verbatim}
    \end{itemize}
\end{itemize}

\subsection*{8. Sets}
\begin{itemize}
    \item Unordered collection of unique elements.
    \item Syntax: \texttt{set = {element1, element2, element3}}
    \item Example:
    \begin{verbatim}
my_set = {1, 2, 3, 3}
print(my_set)  # {1, 2, 3}  # Duplicate 3 is removed

# Adding and removing elements
my_set.add(4)
my_set.remove(2)
print(my_set)  # {1, 3, 4}
    \end{verbatim}
    \item Methods:
    \begin{itemize}
        \item \textbf{\texttt{.add()}}: Add an element.
        \begin{verbatim}
        s = {1, 2}
        s.add(3)
        s  # Output: {1, 2, 3}
        \end{verbatim}
        \item \textbf{\texttt{.remove()}}: Remove an element.
        \begin{verbatim}
        s = {1, 2, 3}
        s.remove(2)
        s  # Output: {1, 3}
        \end{verbatim}
        \item \textbf{\texttt{.union()}}: Combine two sets.
        \begin{verbatim}
        s1 = {1, 2}
        s2 = {2, 3}
        s1.union(s2)  # Output: {1, 2, 3}
        \end{verbatim}
        \item \textbf{\texttt{.intersection()}}: Find common elements.
        \begin{verbatim}
        s1 = {1, 2}
        s2 = {2, 3}
        s1.intersection(s2)  # Output: {2}
        \end{verbatim}
    \end{itemize}
\end{itemize}

\subsection*{9. Dictionaries}
\begin{itemize}
    \item Unordered collection of key-value pairs.
    \item Syntax: \texttt{dict = {key1: value1, key2: value2}}
    \item Example:
    \begin{verbatim}
my_dict = {'a': 1, 'b': 2}
print(my_dict['a'])  # 1
my_dict['c'] = 3  # Adds new key-value pair
print(my_dict)  # {'a': 1, 'b': 2, 'c': 3}

# Loop through dictionary keys and values
for key, value in my_dict.items():
    print(key, value)  # a 1, b 2, c 3
    \end{verbatim}
    \item Methods:
    \begin{itemize}
        \item \textbf{\texttt{.get()}}: Retrieve a value, with an optional default.
        \begin{verbatim}
        d = {'a': 1, 'b': 2}
        d.get('a')  # Output: 1
        d.get('c', 0)  # Output: 0
        \end{verbatim}
        \item \textbf{\texttt{.keys()}}: Get all keys.
        \begin{verbatim}
        d = {'a': 1, 'b': 2}
        list(d.keys())  # Output: ['a', 'b']
        \end{verbatim}
        \item \textbf{\texttt{.values()}}: Get all values.
        \begin{verbatim}
        d = {'a': 1, 'b': 2}
        list(d.values())  # Output: [1, 2]
        \end{verbatim}
        \item \textbf{\texttt{.items()}}: Get all key-value pairs as tuples.
        \begin{verbatim}
        d = {'a': 1, 'b': 2}
        list(d.items())  # Output: [('a', 1), ('b', 2)]
        \end{verbatim}
        \item \textbf{\texttt{.pop()}}: Remove a key and return its value.
        \begin{verbatim}
        d = {'a': 1, 'b': 2}
        value = d.pop('a')  # Removes 'a' and returns 1
        d  # Output: {'b': 2}
        \end{verbatim}
        \item \textbf{\texttt{.popitem()}}: Remove and return the last inserted key-value pair as a tuple.
        \begin{verbatim}
        d = {'a': 1, 'b': 2}
        pair = d.popitem()  # Removes and returns ('b', 2)
        d  # Output: {'a': 1}
        \end{verbatim}
        \item \textbf{\texttt{.clear()}}: Remove all items from the dictionary.
        \begin{verbatim}
        d = {'a': 1, 'b': 2}
        d.clear()
        d  # Output: {}
        \end{verbatim}
    \end{itemize}
\end{itemize}

\subsection*{10. Tuples}
\begin{itemize}
    \item Immutable ordered collection of elements.
    \item Syntax: \texttt{tuple = (element1, element2, element3)}
    \item Example:
    \begin{verbatim}
my_tuple = (1, 2, 3)
print(my_tuple[1])  # 2  # Accessing by index

# Unpacking a tuple
x, y, z = my_tuple
print(x, y, z)  # 1 2 3
    \end{verbatim}
    \item Methods:
    \begin{itemize}
        \item \textbf{\texttt{.count()}}: Count occurrences of an element.
        \begin{verbatim}
        t = (1, 2, 2, 3)
        t.count(2)  # Output: 2
        \end{verbatim}
        \item \textbf{\texttt{.index()}}: Find the index of an element.
        \begin{verbatim}
        t = (1, 2, 3)
        t.index(3)  # Output: 2
        \end{verbatim}
    \end{itemize}
\end{itemize}

\subsection*{11. Strings}
\begin{itemize}
    \item A sequence of characters.
    \item Common operations:
    \begin{itemize}
        \item \texttt{len(string)}: Length of string.
        \item \texttt{string.upper()} or \texttt{string.lower()}: Convert to uppercase or lowercase.
        \item \texttt{string.split()}: Split string into list of words.
        \item \texttt{string.replace(old, new)}: Replace substrings.
    \end{itemize}
    \item Example:
    \begin{verbatim}
my_string = "Hello, world!"
    \end{verbatim}
    \item Methods
\begin{itemize}
    \item \textbf{\texttt{.count()}}: Count occurrences of a substring.
    \begin{verbatim}
my_string = "hello world"
my_string.count("l")  # Output: 3
    \end{verbatim}

    \item \textbf{\texttt{.find()}}: Find the index of the first occurrence of a substring.
    \begin{verbatim}
my_string = "hello world"
my_string.find("o")  # Output: 4
    \end{verbatim}

    \item \textbf{\texttt{.index()}}: Find the index of the first occurrence of a substring (raises error if not found).
    \begin{verbatim}
my_string = "hello world"
my_string.index("o")  # Output: 4
    \end{verbatim}

    \item \textbf{\texttt{.lower()}}: Convert all characters to lowercase.
    \begin{verbatim}
my_string = "HELLO"
my_string.lower()  # Output: "hello"
    \end{verbatim}

    \item \textbf{\texttt{.upper()}}: Convert all characters to uppercase.
    \begin{verbatim}
my_string = "hello"
my_string.upper()  # Output: "HELLO"
    \end{verbatim}

    \item \textbf{\texttt{.replace()}}: Replace a substring with another substring.
    \begin{verbatim}
my_string = "hello world"
my_string.replace("world", "Python")  
        # Output: "hello Python"
    \end{verbatim}

    \item \textbf{\texttt{.split()}}: Split a string into a list of substrings.
    \begin{verbatim}
my_string = "hello world"
my_string.split()  # Output: ['hello', 'world']
    \end{verbatim}

    \item \textbf{\texttt{.strip()}}: Remove leading and trailing spaces.
    \begin{verbatim}
my_string = "   hello world   "
my_string.strip()  # Output: "hello world"
    \end{verbatim}

    \item \textbf{\texttt{.join()}}: Join elements of an iterable (e.g., list) into a string.
    \begin{verbatim}
my_list = ['hello', 'world']
" ".join(my_list)  # Output: "hello world"
    \end{verbatim}

    \item \textbf{\texttt{.startswith()}}: Check if the string starts with a specified substring.
    \begin{verbatim}
my_string = "hello world"
my_string.startswith("hello")  # Output: True
    \end{verbatim}

    \item \textbf{\texttt{.endswith()}}: Check if the string ends with a specified substring.
    \begin{verbatim}
my_string = "hello world"
my_string.endswith("world")  # Output: True
    \end{verbatim}
\end{itemize}

\end{itemize}
